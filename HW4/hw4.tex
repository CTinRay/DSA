\documentclass[fleqn,a4paper,12pt]{article}
\usepackage[top=1in, bottom=1in, left=1in, right=1in]{geometry}



\title{DSA Homework \#4}
\date{}

\setcounter{section}{4}

\usepackage{listings}

\usepackage{amsmath}
\usepackage{amssymb}

\usepackage{qtree}

% \usepackage{fontspec}
% \setmainfont{FreeSans}
%\usepackage{unicode-math}

\usepackage{mathspec}
\setmainfont{FreeSans}
\setmathsfont(Digits,Latin,Greek)[Numbers={Lining,Proportional}]{FreeSerif}
\newfontfamily\ZhFont{文泉驛微米黑}
\newfontfamily\SmallFont[Scale=0.7]{FreeSans}
\newfontfamily\SmallSmallFont[Scale=0.7]{FreeSans}
\usepackage{fancyhdr}
\usepackage{lastpage}
\pagestyle{fancy}
\fancyhf{}
\rhead{B03902072\ZhFont{江廷睿}}
\lhead{DSA Homework 4}
\rfoot{\thepage / \pageref{LastPage}}


\begin{document}
\maketitle
\thispagestyle{fancy}
\subsection{Trees}
\begin{enumerate}
\item
  %\Tree [.S a [.NP {\bf b} c ] d ]
  \Tree [.E [.X [.A M F ] U ] N ]

\item 
  In the tree $T$, there are $n$ nodes $a_1,a_2,a_3,...,a_n$.\\
  \textbf{Claim:}for every node $a_i ( 0 < i < n )$ , $f(a_i) \leq 2^n - 1 $.
  \begin{itemize}
    \item
      For $n = 1$:\\
      The only element must be the root element. From the definition given in page 295, its level numbering is $1$.\\
      Because $1 \leq 2^1 - 1$, the claim is true for $n = 1$.\\

    \item
      Assume for $n = k$, $f(a_i) \leq 2^n - 1\ (0 < i < n)$\\

    \item
      For $n = k + 1 $,  \\
      \[ f(a_{k+1}) = \left\{
          \begin{array}{l l}
            f(a_{k+1}) = 2f( a_{i} ) &\quad\text{if v is the left child of node $a_i$}\\
            f(a_{k+1}) = 2f( a_{i} ) + 1 &\quad\text{if v is the right child of node $a_i$} 
          \end{array}
        \right.
      \]
      \begin{align*}
        a_{k+1} &\leq 2f( a_{i} ) + 1 \\
                &\leq 2( 2^n - 1 ) + 1 \\
                &= 2^{n+1} -1               
      \end{align*}
      Therefore, the claim is true when $k = n + 1$.      
  \end{itemize}
  From mathematical induction, it prove that for every node $a_i ( 0 < i < n )$ , \\$f(a_i) \leq 2^n - 1 $.\\
  That is, for every node $v$ of $T$ , $f(v) \leq 2^n - 1$.

\item
  Because in a postorder travelsal, the node $v$ is traveled before all its ascendant and after all its descendant. In a preorder travelsal, the node $v$ is traveled before all its descendant and after all its ascendant. Therefore, we can derive the formula:\\
  $post(v) = pre(v) - depth(v) + desc(v) + 1$ 

\end{enumerate}
\end{document}
